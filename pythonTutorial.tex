
% -----------------------------------------------------------------------
% Python Tutorial
% WS 18/19
% -----------------------------------------------------------------------
\documentclass[enabledeprecatedfontcommands, fontsize=12pt,
     open=right, a4paper,
     twoside, DIV=11,
     abstractoff,
     headsepline,
     numbers=noenddot,
     BCOR=15mm,
     headings=standardclasses,
     headings=big]{scrbook}
\KOMAoptions{cleardoublepage=empty}
% Header f�r Variablen
% Variablen
\newcommand{\theSemester}{Sommersemester 2019}
% Variable f�r die Vorlesung
\newcommand{\theProject}{Informatik studieren in der digitalen Gesellschaft\\Collaborative Writing}
% Variable f�r den Studiengang
\newcommand{\theTitle}{Python}
% Variable f�r den Hochschul-Namen
\newcommand{\theSchool}{Hochschule Kaiserslautern}
% Variable f�r den Dozenten
\newcommand{\theAuthor}{
    Julian  Bernhart,
    Manfred Brill,
    Eric Brunk,
    Mathias Fedder,
    Robin Marc Guth,
    Rainer Haffner,
    Matthias Haselmaier,
    Kathrin Hentschel,
    Fabian Kalweit,
    Kevin Konrad,
    Philipp Lauer,
    Miriam Lohm�ller,
    Pascal Pries,
    Anatoli Sch�fer,
    Denis Schlusche,
    Christoph Seibel,
    Marc Zintel
}

%
% ----------------------------------------------------------------------------------------------
% Vorlage f�r Dokumentation auf LaTeX-Basis im Projekt InfoStuDi
% --------------------------------------------------------------------------------------------
\usepackage{mb}
 \usepackage{mbmath}
\usepackage{textcomp}
% Array-Paket f�r mehr Kontrolle der Tabellen
\usepackage{array}
% ifthen f�r Ein- und Ausblenden der L�sungen.
\usepackage{ifthen}
% Paket f�r Postscript Pi-Fonts
\usepackage{pifont}
% PSTricks
%\usepackage{pstricks}
% Paket f�r das einfache Umdefinieren von Listen
\usepackage[shortlabels]{enumitem}
% Alphabetischer Index
\usepackage{imakeidx}
\makeindex[title=Index,columns=2,options=-s german,intoc]
% Header und Footer mit KomaScript
\usepackage[automark, headsepline]{scrlayer-scrpage}
% Caption unteraderm, dass \ref nicht nur zur Caption sondern auch zur Figure springt
\usepackage{caption}
%
% Header f�r Werkzeuge, Begriffe aus dem Software-Management
\input{variablen}
% Farben
\input{colors}
% Standardverzeichnis f�r das Basisverzeichnis der Bilder
%
\newcommand{\imagePath}{./images}
%
\raggedbottom
\setlength{\parskip}{2.0ex}
\setlength{\parindent}{0.0cm}
%% Verhindert Schusterjungen und Hurenkinder
\clubpenalty = 10000
\widowpenalty = 10000 \displaywidowpenalty = 10000
%
\setcounter{secnumdepth}{2}
\setcounter{tocdepth}{2}
%
\pagestyle{scrheadings}
\clearscrheadfoot
% Thema des Dokuments in die Kopfzeile
\ihead{\headmark}
\ohead[]{\pagemark}
\chead{}
\pagestyle{scrheadings}
% Abst�nde zwischen Caption und Bild/Tabelle
\setlength\abovecaptionskip          {0.4em}
\setlength\belowcaptionskip          {0.2em}
% Anteil der Grafiken h�her auf jeder Seite!
\renewcommand{\textfraction}{0.001}
\renewcommand{\topfraction}{0.99}
% Literatur-Stil
\bibliographystyle{geralpha}
% Listingspaket
\usepackage{listings}
\lstloadlanguages{python}
\lstset{language=python}
\definecolor{lstback}{gray}{0.85}
\lstset{backgroundcolor=\color{lstback}}
\lstset{extendedchars=true}
\lstset{showstringspaces = false}
\lstset{basicstyle = \ttfamily \small}
%% listings mit listings.sty
%
% Kommando f�r den Flattersatz bei nebeneinander liegenden Abbildungen
\newcommand{\flatter}{\setlength{\rightskip}{0pt plus 2cm}}
%
% Schritte in einer Aufz�hlung, daf�r einen Z�hler (schritt) und die Umgebung
% schritte definieren.
\newcounter{schritt}
\newenvironment{schritte}%
{\begin{list}%
{Schritt \arabic{schritt}:}%
{\usecounter{schritt}\settowidth{\labelwidth}{Schritt 1:}%
\setlength{\leftmargin}{\labelwidth}\addtolength\leftmargin{\labelsep}%
\parsep0.0ex\partopsep-0.3ex\itemsep2pt\topsep0.0ex}}{\end{list}}
% Dateinamen f�r interne Musterl�sungen
\newcommand{\filename}[1]{%
\ifthenelse{\boolean{solutions}}{\framebox[50mm]{\parbox{40mm}{\textbf #1}}\vspace{6pt}}{}}
%   Taste
\newcommand{\taste}[1]{\small\textsf{#1}\normalsize}
%   gesch�tzte Namen (NICHT in chapter, section usw. verwenden!)
\newcommand{\name}[1]{\textsl{#1}}
%   Fachbegriffe/Erkl�rung von Abk�rzungen
\newcommand{\begriff}[1]{\index{#1}\textit{#1}}
%
\newcommand{\algorithmus}[2]{%
\vspace{4pt}\fboxsep 1mm \framebox[140mm]
{\parbox{135mm}{{\textbf{#1}}\vspace{2pt}#2}}\vspace{4pt}}
%
%   Dateinamen/Pfade
\newcommand{\datei}[1]{\texttt{#1}\normalsize}
%   Tip (in einer Box)
\newcommand{\tip}[1]{%
\vspace{4pt}\fboxsep 1mm \framebox[140mm]
{\parbox{135mm}{{\textbf{Tipp}}:\\ #1}}\vspace{4pt}}

%Achtung in einer Box
\newcommand{\warning}[1]{%
    \par\vspace{4pt}\fboxsep 1mm \framebox[140mm]%
    {\parbox{135mm}{{\textbf{Achtung}}:\\ #1}}\vspace{4pt}%
}
%

%
%   Reflektion (in einer Box)
%
\newcommand{\reflection}[1]{
\begin{quote}\fboxsep 3mm\framebox[140mm][c]{\parbox{130mm}{{\textbf{Reflektion}:\\}#1}}\end{quote}}
%
%   Angabe, was die Studierenden lesen sollen (in einer Box)

\newcommand{\lesen}[1]{%
\begin{minipage}[c]{2.5cm}
\centering
\includegraphics[width=2cm]{\imagePath/Misc/buchicon}%
\end{minipage}%
\begin{minipage}[t]{12cm}%
#1%
\end{minipage}}
%
%   Angabe im Begleittext, was die Studierenden lesen sollen (in einer Box)
%   Dieses Icon wird f�r Angaben verwendet, die nicht verpflichtend zu lesen
%   sind. Also "nice-to-have", "wenn sie noch Zeit haben".
\newcommand{\vertiefen}[1]{%
\begin{minipage}[b]{2.5cm}
\centering
\includegraphics[width=2cm]{\imagePath/Misc/reading}%
\end{minipage}%
\begin{minipage}[b\newcommand{\firefox}{\texttt{Mozilla Firefox}}
\newcommand{\kde}{\texttt{KDE}}]{12cm}%
#1%
\end{minipage}}
%
% Jetzt kommt die Definition der Kontrollfrage; die Nummerierung
% f�r das ganze erhalten wir mit Hilfe von \item{\kontroll}.
%
% Ein Counter f�r die Kontrollfragen. Wir z�hlen diesen Counter selbst hoch
% mit einem entsprechenden Kommando, das wir in \item verwenden.
% 1
\newcounter{kontrollCounter}[section]
\newcommand{\kontroll}[0]{
    \refstepcounter{kontrollCounter}
    % Kapitelnummer.Z�hler f�r das Item
    \arabic{chapter}.\arabic{kontrollCounter}
    % Das Label ist bis auf weiteres "kontrolle:Kapitelnummer:kontrollcounter
    \label{kontrolle:\arabic{chapter}:\arabic{kontrollCounter}}
}
% Jetzt kommt die Definition der Kontrollfrage; die Nummerierung
% f�r das ganze erhalten wir mit Hilfe von \item{\kontroll}.
\newcommand{\kontrollfrage}[1]{%
\begin{minipage}[c]{1.85cm}
\huge{\ding{46}}%
\end{minipage}%
\begin{minipage}[t]{12cm}%
\begin{itemize}#1%
\end{itemize}%
\end{minipage}}
%
% Einblenden von Musterl�sungen
%
% Schalter f�r das ein- und ausblenden der L�sungen
\newboolean{solutions}
%
% Theorem-Umgebung f�r die �bungsaufgaben
% Wichtig: alle Attribute einstellen, dann die neue
% theorem-Umgebung mit newtheorem definieren!
% Oder, wie hier, durch das Einschlie�en in {}
%
{
% Zeilenumbruch bei Aufgaben-�berschrift
\theoremstyle{break}
% "Normaler" Font im Text
\theorembodyfont{\normalfont}
% Kapitelweise neu nummerieren
% 2
\newtheorem{auftitle}{Aufgabe}[section]
}
%
% Definition f�r die Kennzeichnung der �bungsaufgaben
%
\newcommand{\uebung}{%
\vspace*{11pt}%
\begin{tabular}{@{}p{2.25cm}@{}p{11.7cm}}%
\huge{\ding{45}}&\Large{\textbf{�bungsaufgaben}}%
\end{tabular}%
}
% Ein Counter f�r die �bungsaufgaben. Wir z�hlen diesen Counter selbst hoch
% mit einem entsprechenden Kommando, das wir in \item verwenden.
% 3
\newcounter{aufgabenCounter}[section]
\newcommand{\auf}[0]{
    \refstepcounter{aufgabenCounter}
    % Kapitelnummer.Z�hler f�r das Item
    \arabic{chapter}.\arabic{aufgabenCounter}
}
% Der Text der Aufgaben steht im Ordner ./exercises/tasks/aufgabenstellungen,
% so k�nnen wir die Dateien aus der Veranstaltung
% Wahrscheinlichkeitsrechnung und Statistik verwenden; und die neuen
% Aufgaben gehen in den allgemeinen Fundus ein.
%

% Umgebungen f�r Satz, Definition, Beweis, ... . Wir orientieren uns am Mathematik-Buch,
% dort gab es diese Umgebungen auch schon. Im Grunde sind das einfach
% wieder theorem-Umgebungen.
{
\setlength\theorempreskipamount{5pt plus 3pt minus 1.5pt}
\setlength\theorempostskipamount{5pt plus 1pt minus 1pt}
% "Normaler" Font im Text
\theorembodyfont{\normalfont}
% Die Namen sind so gew�hlt, dass sie kompatibel zu Beamer sind; dann k�nnen
% wir den Text aus Folien kopieren und umgekehrt auch.
% 4
\newtheorem{Satz}{Satz}[section]
\newtheorem{Fakt}{Fakt}[section]
\newtheorem{Definition}{Definition}[section]
}
%
{
\setlength\theorempreskipamount{8pt plus 3pt minus 1.5pt}
\setlength\theorempostskipamount{5pt plus 1pt minus 1pt}
\theoremstyle{break}
% "Normaler" Font im Text
\theorembodyfont{\normalfont}
\newtheorem{datensatz}{Datensatz}
}
% Text mit Pfad
\newcommand{\aufgabentext}[1]{\renewcommand{\labelenumi}{\alph{enumi})}\auftitle\label{#1}\input{./exercises/tasks/#1} }
% Funktion f�r die L�sungs-Hinweise f�r eine Aufgabe. Der Text
% steht analog zu den �bungsaufgaben im Ordner tasks/solutions.
\newcommand{\hinweistext}[1]{\renewcommand{\labelenumi}{\alph{enumi})}\input{./exercises/solutions/#1}}
% Und jetzt die Funktionen, die wir im Text aufrufen
\newcommand{\aufgabe}[1]{\aufgabentext{#1}}
% L�sungshinweise im Anhang
\newcommand{\hinweis}[1]{\subsubsection*{Aufgabe \ref{#1}}\label{#1sol}  \hinweistext{#1}}
%
% Datensatz-Texte und Daten in eigener Datei
\newcommand{\dataset}[1]{\begin{datensatz}\label{#1}\input{./exercises/datasets/#1}\end{datensatz}}
% Teilaufgaben alphabetisch nummerieren
\renewcommand{\labelenumi}{\arabic{enumi})}
%
% Marginalien
%
\newcommand{\randnotiz}[1]{\marginpar{\small{\textbf{#1}}}}
%
% Gabelschl�ssel als Marginalie und Hinweis auf Praxisbezug
\newcommand{\praxisbezug}[0]{\randnotiz{\includegraphics[width=1cm]{\imagePath/Misc/gabel}}}
%
% alert aus beamer-Folien zu emph machen
%
\newcommand{\alert}[1]{\emph{#1}}
%
%
\newcommand{\titelseite}[1]{%
% Titelseite
\pagenumbering{roman}%
\thispagestyle{empty}%
\begin{titlepage}%
% Volle Zeilenbreite verwenden!
\centering%
\vspace*{2cm}%

\Huge{\textbf{#1}}\\\vspace*{0.5cm}%

\vspace*{10cm}%

\Large{\theAuthor{}}%

\Large{\theProject{}}%

\Large{\theSchool{}}%
\end{titlepage}}

% Titelseite mit zus�tzlicher Grafik
%
% Das obligatorische Argument ist der Titel des Dokuments.
% Als Default wird das Logo der Stochastik-Veranstaltung als Titelbild
% verwendet. Mit Hilfe eines optionalen Arguments kann
% ein anderes Bild verwendet werden!
% Beispiel: \titelseite[\imagePath/misc/ameise]{Das Liebesleben der Ameisen}
% Aufruf mit Standardbild:
% \titelseite{Das Liebesleben der Ameisen}
%
\newcommand{\titelseiteMitBild}[2][\imagePath/logos/python]{%
%% Titelseite
\pagenumbering{roman}%
\thispagestyle{empty}%
\begin{titlepage}%
% Volle Zeilenbreite verwenden!
\centering%
\vspace*{0.5cm}%

\Huge{\textbf{#2}}\\\vspace*{0.5cm}%

\vspace*{2.0cm}%
\includegraphics[height=6cm]{#1}%

\vspace*{1.0cm}%

\Large{\theProject{}}%

\Large{\theSchool{}}%

\vspace*{2.0cm}%

\normalsize{\theAuthor{}}%
\end{titlepage}}

%\makeatletter
%\newcommand{\shoppinglist}[1]{%
%   Shopping list: #1\checknextarg}
%\newcommand{\checknextarg}{\@ifnextchar\bgroup{\gobblenextarg}{ and that's all!}}
%\newcommand{\gobblenextarg}[1]{ and also #1\@ifnextchar\bgroup{\gobblenextarg}{ and that's all!}}
%\makeatother

\makeatletter
\newcommand{\templabel}{xxx}
\newcommand{\uebungTutorial}[1]{%
    \vspace*{11pt}%
    \begin{tabular}{@{}p{2.25cm}@{}p{11.7cm}}%
        \huge{\ding{45}}&\Large{\textbf{�bungsaufgaben}}%
    \end{tabular}%
    \aufgabe{#1}
    \renewcommand{\templabel}{#1sol}
    \checknextarg{#1}
}
\newcommand{\checknextarg}[1]{\@ifnextchar\bgroup{\gobblenextarg}{ \leavevmode \\ \\ Die L�sungen zu den Aufgaben finden Sie im Anhang \ref{#1sol}. }}
\newcommand{\gobblenextarg}[1]{  \aufgabe{#1}\@ifnextchar\bgroup{\gobblenextarg}{ \leavevmode \\ \\  Die L�sungen zu den Aufgaben finden Sie im Anhang \ref{\templabel}.}}
\makeatother

%
% Schalter f�r das Ein- und Ausblenden der L�sungen
\setboolean{solutions}{true}
%
\input{pdfsetup}
\listfiles
%
% Beginn Dokument
%
\begin{document}
\titelseiteMitBild{\theTitle{}}
%
% Vorwort
%
% !TeX root = ../pythonTutorial.tex
\chapter*{Vorwort}

Das vom Stifterverband gef�rderte Projekt \glqq Informatik studieren in der digitalen Gesellschaft (InfoStuDi)\grqq{}
erprobt und evaluiert neue Lehr-, Lern- und Pr�fungsformen in den Informatik-Studieng�ngen im Fachbereich Informatik und Mikrosystemtechnik der Hochschule Kaiserslautern.

Studieng�nge an einer Hochschule f�r
angewandte Wissenschaften bereiten die Studierenden auf die sp�tere Arbeitswelt vor.
Diese Arbeitswelt
wird von zeitlich und �rtlich ungebundenem T�tigkeiten gepr�gt sein.
Im Teilprojekt \glqq Collaborative Writing\grqq{} wurde eine neue Form einer Lehrveranstaltung erprobt,
die die Studierenden auf diese sp�tere Arbeitswelt vorbereiten soll.
Ein Team aus Studierenden und Lehrenden verfasst ein Dokument zu einem Thema der Informatik.
Dabei wird neben der Produktion von Texten auch Software entstehen.
Die Produktion des vorliegenden Dokuments
zum Thema Python wurde wie ein gro�es agiles Software-Projekt organisiert.
Drei Sprints wurden durchgef�hrt, das Team organisierte sich selbst. Werkzeuge wie \LaTeX{}, Git oder Jenkins wurden eingesetzt.
Die Studierenden waren nicht nur Autoren, sondern auch Fachlektoren, Software-Entwickler und f�r die Qualit�t des Gesamtergebnisses mit verantwortlich.

Dieses Projekt w�re nicht zustande gekommen ohne die Studierenden, die sich auf dieses Abenteuer im Rahmen der Lehrveranstaltung \glqq Aktuelle Themen aus Forschung und Praxis\grqq{} des Masterstudiengangs Informatik im Wintersemester~2018/19 eingelassen haben. An dieser Stelle ein herzliches \glqq Danke sch�n!\grqq{} f�r das Vertrauen und den Mut, sich auf diese Form einer
Lehrveranstaltung einzulassen.
Miriam Lohm�ller, als wissenschaftliche Mitarbeiterin im Projekt InfoStuDi t�tig, brachte ihre Erfahrung aus dem Verlagswesen ein und hat die von den Studierenden verfassten
Texte lektoriert. Fabian Kalweit, Mitarbeiter des Projektleiters im Fachbereich Informatik und Mikrosystemtechnik, hat das fachliche Lektorat unterst�tzt und insbesondere das Backend in GitHub
organisiert und gestaltet.

Der vorliegende Text stellt den Stand im M�rz 2019, nach Abschluss der Lehrveranstaltung, dar. Nat�rlich
ist das Python-Tutorial nicht abgeschlossen. Das komplette Projekt steht in Form eines �ffentlichen Git-Repositories (\cite{githubRepo}) 
zur Verf�gung und kann von interessierten Studierenden verwendet und vor allem weiterentwickelt werden.
Alle Autoren hoffen, dass unsere Leser den Text f�r gut befinden.

Die Texte wurden nach bestem Wissen und
Gewissen verfasst. Sollte der Text trotzdem Fehler enthalten, liegen diese in der alleinigen Verantwortung
des Projektleiters!

\vspace{\baselineskip}
\begin{flushright}\noindent
Zweibr�cken, im M�rz 2019\hfill

\hfill {\it Manfred  Brill}
\end{flushright}
\pagebreak
%\section*{Das Team}
%Gruppenbild mit Dame -- das Team nach dem letzten Sprint Meeting am 31.~Januar~2019.

\begin{figure}[ht]
\centering
\includegraphics[width=\textwidth]{images/theTeam}% Rotieren mit rotate=90
%\caption{\label{vorwort:team}Das Projektteam}
\end{figure}
\begin{tabular}{ll}
\centering
&Das Projektteam nach dem letzten Sprint Meeting\\
&von links nach rechts:\\
Hintere Reihe&Fabian Kalweit, Matthias Haselmaier, Marc Zintel,\\
                   &Robin Guth, Anatoli Sch�fer, Denis Schlusche,\\
                   &Kevin Konrad, Miriam Lohm�ller \\
Vordere Reihe&Mathias Fedder, Rainer Haffner, Lukas Kuhn,\\
                   &Sebastian Morsch, Julian Bernhart, Phillip Lauer,\\
                   &Christoph Seibel\\
Ganz vorne&Manfred Brill
\end{tabular}

%
% Inhaltsverzeichnis
%
\tableofcontents
\clearevenpage
\pagenumbering{arabic}
%
%
%
\input{./chapters/grundlagen.tex}

\input{./chapters/inputOutput.tex}

\input{./chapters/functionsAndModules.tex}


\input{./chapters/testing.tex}

\input{./chapters/ui.tex}

\input{./chapters/bibliotheken.tex}

% !TeX root = ../../pythonTutorial.tex

\chapter{Dokumentation}
\label{documentation:sec:Dokumentation}

Wie in allen Programmiersprachen ist vor allem bei gro�en Programmen eine gute Dokumentation wichtig.
Viele Funktionen sind ohne ausreichende Beschreibungen nicht gut nachvollziehbar und man kann oft nur erahnen, was ein bestimmter Programmteil letztendlich bewirkt.
Da von vielen Programmierern aus Bequemlichkeit und Schreibfaulheit auf eine umfassende Dokumentation verzichtet wird, sind viele Programme kaum bis �berhaupt nicht wartbar oder ver�nderbar.

In Python gibt es f�r genau dieses Problem Tools, mit dessen Hilfe die Dokumentation von Programmen auf ein angenehmes Ma� reduziert wird.
Eines dieser plattformunabh�ngigen Werkzeuge mit dem Namen \textit{epydoc} wird im Folgenden genauer betrachtet.
\randnotiz{epydoc} epydoc analysiert Python-Programme anhand des Quellcodes, verwertet diesen und gibt anschlie�end eine Dokumentation als PDF- oder HTML-Datei aus.
Bei dieser Analyse werden Doc- strings ben�tigt, deren Benutzung und Einbindung noch genauer betrachtet wird.

Unter \url{http://epydoc.sourceforge.net} k�nnen Sie sich die aktuelle epydoc-Version herunterladen und anschlie�end installieren.


\section{Epydoc}
\label{documentation:sec:epydoc}

Nach der erfolgreichen Installation kann epydoc �ber die Konsole aufgerufen werden.
Dazu wird der Befehl \textit{epydoc.py} benutzt. (in Linux ohne .py)
Eine Dokumentation zu einer bestimmten Python-Datei ist dann m�glich, wenn epydoc an der selben Stelle im Dateipfad ausgef�hrt wird.

Um das Format zwischen PDF und HTML zu wechseln, wird der Befehl \textit{--pdf} beziehungsweise \textit{--html} verwendet.
Auch den Speicherort der Dokumentation kann frei variiert werden.
Dazu wird der Befehl \textit{--output} vor den gew�nschten Verzeichnisort geschrieben.

Ein Beispiel f�r eine korrekte Dokumentationserstellung durch epydoc sieht wie folgt aus:
\begin{lstlisting}[label=documentation:lst:epydocbsp]
$ epydoc.py --html --output eigenes_verzeichnis testprogramm
\end{lstlisting}
Dadurch wird die Dokumentation zum Modul \textit{testprogramm} als HTML-Datei im Verzeichnis \textit{gewuenschtesverzeichnis} gespeichert.


\section{Docstrings}
\label{documentation:sec:docstrings}

Python bietet uns durch drei einfache oder doppelte Anf�hrungszeichen die M�glichkeit, sogenannte Blockkommentare �ber mehrere Zeilen zu verfassen.
Der darin eingefasste Text wird als Documentation String, kurz \textit{Docstring} bezeichnet.

\begin{lstlisting}[label=documentation:lst:docstringbsp]
"""
Dieser Text ist ein Docstring.
Er wird von epydoc als solcher erkannt,
analysiert und interpretiert.
"""
\end{lstlisting}

Docstrings sollten zur Beschreibung aller Programmierteile benutzt werden, vor allem Funktionen und Klassen sollten durchgehend kommentiert werden.

\begin{lstlisting}[label=documentation:lst:docstringbsp2]
class myclass(object):
   """Docstring zur Klasse myclass
   Beschreibung, wozu die Klasse dient.
   """

def myfunction():
   """Docstring zur Funktion myfunction
   Beschreibung, was die Funktion macht.
   """
\end{lstlisting}

Innerhalb des Programms kann man durch einen Befehl einen beliebigen Docstring aufrufen.
Dazu dient das Attribut \_doc\_, das zu jeder Instanz automatisch erstellt wird.

\begin{lstlisting}[label=documentation:lst:docstringausgeben]
>>> print myclass.__doc__
Docstring zur Klasse myclass
   Beschreibung, wozu die Klasse dient.
\end{lstlisting}


Python hat intern jedoch keine M�glichkeit, um f�r Variablen Docstrings zu nutzen.
Diese M�glichkeit wird durch epydoc erm�glicht.
Folgt in der folgenden Zeile nach einer Variablenzuweisung ein Blockkommentar, wird dieser automatisch als Docstring zur Variablen interpretiert.
Eine weitere M�glichkeit ist es, den Docstring vor die Wertezuweisung der Variablen zu definieren.
In diesem Fall wird auf die dreifachen Anf�hrungszeichen verzichtet und je Zeile eine Raute mit Doppelpunkt \#: vorangestellt.

\begin{lstlisting}[label=documentation:lst:docstringepydoc]
a = 42
""" Die Variable a ist 42
"""

#: Die Variable b
#: ist 24
b = 24
\end{lstlisting}


\section{Epytext}
\label{documentation:sec:Epytext}
Um sicherzustellen, dass epydoc die Docstrings richtig interpretiert, wurde die Beschreibungssprache Epytext eingef�hrt.
Darin sind Regeln f�r die Formatierung und die Umsetzung mit Docstrings festgehalten, um eine einheitliche Benutzung zu gew�hrleisten.
Im Folgenden werden einige der Regeln und M�glichkeiten betrachtet.

Zum einen ist in Epytext, genau wie in Python die korrekte Einr�ckung von entscheidender Wichtigkeit.

\begin{lstlisting}[label=documentation:lst:einrueckung]
def myfunction():
   """
   Wichtig ist die richtige Einr�ckung
   """
\end{lstlisting}

Auch Listen sind innerhalb von Docstrings m�glich.
Diese k�nnen als einfache Auflistung oder  Nummerierung benutzt werden.
\begin{lstlisting}[label=documentation:lst:liste]
"""
Auflistung:
- Eier
- Milch
- Mehl

Nummerierung:
1. Aufstehen
2. Z�hne putzen
3. Duschen
"""
\end{lstlisting}

Weitere Formatierungsm�glichkeiten werden kurz in folgendem Beispiel gezeigt.
Der Aufbau ist dabei stets in der Form eines Gro�buchstabens mit dem zu formatierenden Strings in geschweiften Klammern \textit{x{*}}.

\begin{lstlisting}[label=documentation:lst:liste]
"""
I{Dieser String ist kursiv}
F{Dieser String ist fett}
M{Dies ist ein mathematischer Ausdruck}
U{String ist eine URL und wird als Hyperlink interpretiert}
E{Verhindert die Interpretation}
"""
\end{lstlisting}

\section{Zusammenfassung}
\label{documentation:sec:zusammenfassung}
Vielen Programmierern ist die st�ndige Dokumentation seit jeher ein Dorn im Auge.
Mit epydoc ist in der Python-Programmierung jedoch ein Tool vorhanden, welche einem die Arbeit zu einem gewissen Teil abnimmt.
Deshalb sollte dieses oder ein �hnliches Programm verwendet werden, um sicherzustellen, dass auch nach der Fertigstellung des Programms nachvollziehbar wird, was der eigentliche Sinn und Zweck hinter den einzelnen Programmteilen ist.
Falls Sie sich nach diesem Grund�berblick noch weiter �ber epydoc und Epytext informieren m�chten, k�nnen Sie sich die Dokumentation auf der offiziellen Homepage von epydoc anschauen.

Diese finden Sie unter \url{http//epydoc.sourceforge.net}. 

\input{./chapters/advancedTopics.tex}





%
% Literatur
%
\cleardoublepage
\phantomsection
\addcontentsline{toc}{chapter}{Literaturverzeichnis}
\chaptermark{Literaturverzeichnis}
\sectionmark{Literatur}\label{literatur}
\bibliography{./bib/literatur}
%
% Anhang
%
\appendix

\chapter{L�sungshinweise}\label{solhinweise}
Hier finden sich die L�sungshinweise zu den Aufgaben.

\section{L�sungen zu Grundlagen}
\hinweis{grundlagen01}
\hinweis{grundlagen02}

\section{L�sungen zu Datentypen und Kontrollstrukturen}
\hinweis{DatatypesAufgabe1}
\hinweis{ifelseAufgabe1}
\hinweis{KontrollstrukturenAufgabe1}
\hinweis{exceptionhandling01}
\hinweis{exceptionhandling02}
\hinweis{exceptionhandling03}

\section{L�sungen zu Collections}
\hinweis{Collections/CollectionsAufgabe1List}
\hinweis{Collections/CollectionsAufgabe2List}
\hinweis{Collections/CollectionsAufgabe1Tuple}
\hinweis{Collections/CollectionsAufgabe2Tuple}
\hinweis{Collections/CollectionsAufgabe1Set}
\hinweis{Collections/CollectionsAufgabe1Dictionary}

\section{L�sungen zu Iteratoren}
\hinweis{iterator/iterator01}
\hinweis{iterator/iterator02}

\section{L�sungen zu Testen}
\hinweis{testen01}
\hinweis{testen02}

\section{L�sungen zu Klassen und Objekte}
\hinweis{classesandobjects01}
\hinweis{classesandobjects02}

\section{L�sungen zu Funktionen und Module}
\input{./exercises/solutions/functionsAndModules/exercisesFunctionsAndModules.tex}

\section{L�sungen zur Ein- und Ausgabe}
\input{./exercises/solutions/inputOutput/exercisesInputOutput.tex}

\section{L�sungen zu Benutzeroberfl�chen}
\hinweis{UI_aufgabe01}
\hinweis{UI_aufgabe02}
\hinweis{UI_aufgabe03}

\section{L�sungen zu Bibliotheken}
\hinweis{numpy_01}
\hinweis{numpy_02}

\section{L�sungen zu Maschinelles Lernen}

\input{./exercises/solutions/MachineLearning/exercises_machinelearning.tex}


\section{L�sung zu Datenbanken}
\hinweis{database01}
\hinweis{database02}
\section{L�sungen zu Nebenl�ufigkeit}
\hinweis{nebenlaufigkeit01}
\hinweis{nebenlaufigkeit02}
\hinweis{nebenlaufigkeit03}
\hinweis{nebenlaufigkeit04}
\hinweis{nebenlaufigkeit05}
\hinweis{nebenlaufigkeit06}
\hinweis{nebenlaufigkeit07}
\hinweis{nebenlaufigkeit08}
\hinweis{nebenlaufigkeit09}
\hinweis{nebenlaufigkeit10}
\hinweis{nebenlaufigkeit11}
\hinweis{nebenlaufigkeit12}
\hinweis{nebenlaufigkeit13}
%
% Index
%\clearevenpage
%\phantomsection
%\small
%\chaptermark{Index}
%\sectionmark{Index}
%\printindex
%\normalsize
\end{document}

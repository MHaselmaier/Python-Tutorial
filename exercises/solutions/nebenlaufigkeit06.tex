\begin{enumerate}
\item Kann aus der \hyperref[threads:lst:counter_condition_variable_example]{\lstinline$Counter$-Klasse}
�bernommen werden.

\item Da nicht sichergestellt werden kann, dass durch \lstinline$notify()$ genau der eine Thread 
aufgeweckt wird, dessen Bedingung als n�chstes erf�llt ist, muss \lstinline$notify_all()$ aufgerufen 
werden, nachdem der \lstinline$Counter$ inkrementiert wurde.

\lstinputlisting[language=Python,linerange={1-2,11-19}]{exercises/src/nebenlaufigkeit06.py}

\item Im Konstruktor wird ein neues Attribut \lstinline$digit$ definiert, das mit dem jeweiligen Parameter
initialisiert wird.
Der Aufruf des \lstinline$IncrementerThread$-Konstruktors muss entsprechend angepasst
werden.

\lstinputlisting[language=Python,linerange={1-2,21-26}]{exercises/src/nebenlaufigkeit06.py}

\item Die neue \lstinline$check_condition()$-Methode gibt den Wert des Ausdrucks 
\lstinline$self.counter.count % 10 == self.digit$ zur�ck.

\lstinputlisting[language=Python,linerange={1-2,38-42}]{exercises/src/nebenlaufigkeit06.py}

\item In der \lstinline$for$-Schleife innerhalb von \lstinline$run()$ wird ein \lstinline$with$-Statement 
erg�nzt.
Es beinhaltet die Aufrufe von \lstinline$wait_for()$ und \lstinline$increment()$.
Hierbei bekommt \lstinline$wait_for()$ die \lstinline$check_condition()$-Methode �bergeben.

\lstinputlisting[language=Python,linerange={1-2,27-37}]{exercises/src/nebenlaufigkeit06.py}

\item Eine Ausgabe in \lstinline$check_condition()$ und eine Ausgabe vor \lstinline$increment()$ gen�gen,
um das Geschehen nachzuvollziehen. Es sollten die Werte des \lstinline$Counters$ und des Attibuts
\lstinline$digit$ ausgegeben werden.
\end{enumerate}

Zur Verwendung von Threads muss das \lstinline$threading$-Modul importiert werden.
Zur Verwendung von \lstinline$sleep()$ muss das \lstinline$time$-Modul importiert werden.
Es muss eine Klasse von \lstinline$Thread$ abgeleitet werden.
Diese Klasse muss die \lstinline$run()$-Methode �berschreiben und in einer \lstinline$for$-Schleife, 
welche von 1 bis 10 l�uft, die aktuelle Zahle ausgeben und eine Sekunde \glqq schlafen\grqq{}.

\lstinputlisting[language=Python,linerange={1-2,7-12}]{exercises/src/nebenlaufigkeit01.py}

Es muss ein Objekt dieser Klasse erzeugt und die \lstinline$start()$ aufgerufen werden.

\lstinputlisting[language=Python,linerange={1-2,14-16}]{exercises/src/nebenlaufigkeit01.py}